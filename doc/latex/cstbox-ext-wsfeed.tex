% !TeX encoding = UTF-8
% !TeX spellcheck = en-GB
\documentclass[11pt,a4paper]{article}

\usepackage[utf8]{inputenc}
\usepackage{graphicx}
\usepackage{xspace}
\usepackage{float}
\usepackage{xcolor}
\usepackage{url}
\usepackage[english]{babel}
\usepackage{listings}
\usepackage{hyperref}
\hypersetup{
	hidelinks,
	colorlinks=true,
	linkcolor=darkgray
}
\usepackage{lmodern}
\usepackage[left=3cm,right=3cm,top=3cm,bottom=3cm]{geometry}
\usepackage{vhistory}
\usepackage[bottom]{footmisc}
\usepackage{fancyhdr}
\usepackage{booktabs}
\usepackage{framed}
%\usepackage{lipsum}

%\usepackage{tikz}
%\usetikzlibrary{shapes,shadows,calc}

\newcommand{\cbx}{CSTBox\xspace}
\newcommand{\wsf}{WSFeed\xspace}
\newcommand{\thetitle}{\cbx \wsf module documentation}

\author{Eric PASCUAL}
\title{\thetitle}
\date{Sep. 29, 2015}
\newcommand\rev{0.4}

\newcommand{\apireq}[2]{\vspace{1em}\textbf{Request : }\texttt{#1~#2}}

\newcommand{\reqenvwidth}{\textwidth}

\newenvironment{reqflds} {%
	\vspace{1em}\textbf{Request fields : }\hfill\\ %
	\tabularx{\reqenvwidth}{llX} %
	\textbf{Name} & \textbf{Type} & \textbf{Description}\\ %
	\midrule %
}{ %
	\endtabularx %
}

\newcommand{\reqfld}[3]{\texttt{#1} & #2 & #3 \\}

\newenvironment{reqargs} { %
	\vspace{1em}\textbf{Request arguments : }\hfill\\ %
	\tabularx{\reqenvwidth}{llX} %
	\textbf{Name} & \textbf{Type} & \textbf{Description}\\ %
	\midrule %
}{ %
	\endtabularx %
}

\newcommand{\reqarg}[3]{\texttt{#1} & #2 & #3 \\}

\newcommand{\reqbody}[1]{\vspace{1em}\textbf{Request body : }\texttt{#1}}

\newenvironment{responses} { %
	\vspace{1em}\textbf{Responses : }\hfill\\ %
	\tabularx{\reqenvwidth}{lp{0.5\textwidth}X} %
	\textbf{Status} & \textbf{Reply body} & \textbf{Description}\\ %
	\midrule %
}{ %
	\endtabularx %
}

\newcommand{\response}[3]{#1 & \texttt{#2} & #3 \\}

\setlength\extrarowheight{0.5em}

\renewcommand{\headrulewidth}{0.5pt}
\pagestyle{fancy}
\fancyhf{}
\lhead{\cbx}
\rhead{\wsf module documentation rev.\rev}
\cfoot{\thepage}

\colorlet{punct}{red!60!black}
\definecolor{background}{HTML}{EEEEEE}
\definecolor{delim}{RGB}{20,105,176}
\colorlet{numb}{magenta!60!black}

\lstdefinelanguage{json}{
    basicstyle=\normalfont\ttfamily,
    numbers=left,
    numberstyle=\scriptsize,
    stepnumber=1,
    numbersep=8pt,
    showstringspaces=false,
    breaklines=true,
    frame=lines,
    backgroundcolor=\color{background},
    literate=
     *{0}{{{\color{numb}0}}}{1}
      {1}{{{\color{numb}1}}}{1}
      {2}{{{\color{numb}2}}}{1}
      {3}{{{\color{numb}3}}}{1}
      {4}{{{\color{numb}4}}}{1}
      {5}{{{\color{numb}5}}}{1}
      {6}{{{\color{numb}6}}}{1}
      {7}{{{\color{numb}7}}}{1}
      {8}{{{\color{numb}8}}}{1}
      {9}{{{\color{numb}9}}}{1}
      {:}{{{\color{punct}{:}}}}{1}
      {,}{{{\color{punct}{,}}}}{1}
      {\{}{{{\color{delim}{\{}}}}{1}
      {\}}{{{\color{delim}{\}}}}}{1}
      {[}{{{\color{delim}{[}}}}{1}
      {]}{{{\color{delim}{]}}}}{1},
}

\begin{document}

\maketitle

\begin{abstract}
This document describes the \wsf extension module for the \cbx.
\end{abstract}

\begin{versionhistory}
\vhEntry{0.1}{2015-01-14}{EP}{first draft}
\vhEntry{0.2}{2015-01-19}{EP}{variables definitions management requests added}
\vhEntry{0.3}{2015-02-04}{EP}{typos and minor corrections}
\vhEntry{0.4}{2015-09-29}{EP}{fix wrong service name in URLs}
\end{versionhistory}

\setcounter{table}{0}	% needed to reset counter since version history creates a table

\clearpage
\tableofcontents

\clearpage
\setlength{\parskip}{0.5em}
\setlength{\parindent}{0pt}

\section{Objectives}

The \wsf extension module adds capabilities to the \cbx framework for feeding events 
coming from the external world. These events are injected in event buses as if they were 
produced by embedded extensions modules.

\section{Philosophy}

In order to be as opened as possible, this extension is based on the following principles:
\begin{itemize}
\item the events entry point is provided as a Web Service, 
hence the name of the extension (\wsf stands for "Web Services Feed").
\item time stamping is by default managed by the extension, in order to :
\begin{itemize}
\item simplify at the most what external actors have to provide
\item ensure a coherent time stamping of events in the box realm, since the external feed 
channel can be used in conjunction with other internal ones, such as sensor drivers
\end{itemize}
\end{itemize}

\section{Common considerations} 

\subsection{HTTP methods}

All the services are invoked using HTTP GET, POST, PUT and DELETE methods. 

\subsection{HTTP responses}

\subsubsection{Normal responses}

HTTP response body is used to return the requested data or to provide complementary 
information in case of request processing error. The body content is formatted as a directly 
parsable JSON\footnote{\url{http://www.json.org/}} string. The response header includes the 
appropriate "\texttt{Content-Type: application/json}" field to inform the client.

\subsubsection{Handled error responses}

Handled processing error, such as bad requests, not found resources,\dots are reported using 
the appropriate 4xx status codes. The response body will contain at least the following 
field :

\begin{tabularx}{\textwidth}{lX}
\texttt{message} & short error message\\
\end{tabularx}

In case of syntactically invalid parameters, refering to not existing resources or 
any similar situation, additional information are provided in the \texttt{additInfos} field. 
Its content will depend on the nature of the error. It can be a simple string or a 
dictionary when a structured feedback is useful.

\subsubsection{Unexpected error responses}

Unexpected processing errors on server side are reported with status code 500, the response 
body containing the following fields :

\begin{tabularx}{\textwidth}{lX}
\texttt{errtype} & the name of the error or exception, as defined by the server process\\
\texttt{message} & short error message \\
\texttt{additInfos} & optional additional information\\
\end{tabularx}

\subsection{Authentication}

To be defined

\section{Configuration}

\subsection{Overview}
The configuration of this module consists mainly in declaring the variables which can be fed 
by its services, using the file \texttt{/etc/cstbox/wsfeed.cfg}. 

This file contains a dictionary encoded in JSON format. Keys defined in the current version 
are :

\begin{tabularx}{\textwidth}{lX}
\textbf{Name} & \textbf{Content} \\
\midrule
\texttt{variables} & the definitions of the variables known by the module
\end{tabularx}

Additional top level entries may be added in the future for supporting new needs.

\subsection{Configuration keys detail}

\subsubsection{variables}
\label{sec:cfg-variables}

The associated value is a sub-dictionary containing the definitions of the variables, the 
keys being the names of the variables. The definition of a variable is itself a 
sub-dictionary, using the following keys :
\begin{description}
\item [\texttt{var\_type}]\hfill\\
The semantic type of the variable, in terms of application domain, such as 
\texttt{temperature}, \texttt{current},\dots Valid types are listed hereafter.
\item [\texttt{unit}]\hfill\\
The unit used to express the value, when relevant.
\end{description}

Semantic types currently defined are :
 
\begin{hyphenrules}{nohyphenation}\begin{sloppypar}
\begin{tabularx}{\textwidth}{lX}
\textbf{physical measurements} &
temperature,
volume,
voltage,
current,
frequency,
power,
energy,
phase\_shift,
cos\_phi,
power.react,
energy.react,
flow,
pressure,
static\_pressure,
dynamic\_pressure \\
\textbf{binary states} &
opened,
occupancy,
presence,
flow\_detection,
flood\_detection \\
\textbf{stateless events} & 
detection,
notification \
\end{tabularx}
\end{sloppypar}\end{hyphenrules}

\subsection{Example}

An exemple of configuration file is provided in figure \ref{fig:config-example}.

\begin{figure}[H]
\begin{center}
\begin{lstlisting}[language=json]
{
    "variables": {
        "temp_room": {
            "var_type": "temperature",
            "unit": "degC"
        },
        "phase_1_current": {
            "var_type": "current",
            "unit": "A"
        }
    }
}
\end{lstlisting}
\end{center}
\caption{Example of configuration file}
\label{fig:config-example}
\end{figure}


\section{Web services API}

\subsection{Feed external event}

This request injects one or more events, corresponding to the variable values provided as 
arguments. 

By default, time stamping of the resulting events is generated at broadcast time, using the 
same instant for all of them. If a specific time stamp is needed for a variable, it can be 
provided as a complementary field of its data, in the format described hereafter.

\apireq{POST}{/api/wsfeed/pushval}

\begin{reqargs}
\reqarg{nvt}{variable data}{Name, Value and optional Time stamp}
\end{reqargs}

\textbf{Note:} \texttt{nvt} stands for : \textbf{n}ame, \textbf{v}alue, \textbf{t}ime.

Multiple occurrences of the \texttt{nvt} argument can be used in the request if several 
values are to be fed in at the same time.

The \texttt{nvt} value is formatted as a multiple fields string, separated by a colon (':'). 
These fields are in sequence :
\begin{itemize}
\item the name of the variable 
\item the value of the variable, in standard ISO representation
\item optionally, the time stamp, as a decimal number of seconds since Epoch or as a 
standard ISO 8601 format, including the time zone\footnote{colons which can appear in the 
ISO time stamp do not conflict with fields separators, since the parser takes the rest of 
the value past the second colon as a whole}
\end{itemize}

\begin{responses}
\response{200}{\textit{empty}}{feed processed successfully}

\response{400}{\{\newline
"message": "invalid nvt value",\newline 
"additInfos": \dots\newline
\}}{the provided tuple is invalid (most probable cause : wrong fields count). The 
\texttt{additInfos} field contains the offending argument value}

\response{400}{\{\newline
"message": "invalid value",\newline 
"additInfos": \{\newline
"name": \dots,\newline
"value": \dots\newline
\}\newline
\}}{the provided value is invalid. The \texttt{additInfos} field contains a sub-dictionary 
providing the name of the variable and the offending value}

\response{400}{\{\newline
"message": "invalid timestamp",\newline 
"additInfos": \{\newline
"name": \dots,\newline
"timestamp": \dots\newline
\}\newline
\}}{the provided time stamp is invalid. The \texttt{additInfos} field contains a 
sub-dictionary providing the name of the variable and the offending value}

\response{404}{\{\newline
"message": "undefined variable",\newline 
"additInfos": \dots\newline
\}}
{the variable is not defined. The \texttt{additInfos} field contains the offending variable 
name}
\end{responses}

\subsection{Variable definitions management services}

\subsubsection{Variables definitions retrieval}
\label{sec:vardefs-get}

\apireq{GET}{/api/wsfeed/vardefs}

\begin{responses}
\response{200}{the variables definitions}{formatted as detailed in 
\ref{sec:cfg-variables}}
\end{responses}

\subsubsection{Variables definitions update}
\label{sec:vardefs-post}

\apireq{POST}{/api/wsfeed/vardefs}

\begin{responses}
\response{200}{\textit{empty}}{}
\response{400}{%
\{\newline
"message": "invalid JSON data",\newline 
\}}{provided JSON data are not valid}
\response{400}{%
\{\newline
"message": "invalid variable definition",\newline 
"additInfos": \dots\newline
\}}{provided variable definition is not valid. 
The \texttt{additInfos} field contains the name of the offending variable}
\end{responses}

\end{document}